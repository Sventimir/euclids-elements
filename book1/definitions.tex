\item \bold{Punkt} jest tym, co niepodzielne.
\item \bold{Linia} jest to długość bez szerokości.
\item Linia jest ograniczona punktami.
\item \bold{Prosta} jest to linia, położona równomiernie względem swoich
    punktów.
\item \bold{Powierzchnia} posiada tylko długość i szerokość.
\item Powierzchnia jest ograniczona liniami.
\item \bold{Płaszczyzna} jest to powierzhnia położona równomiernie względem
    swoich punktów.
\item \bold{Kąt płaski} jest to wzajemne nachylenie linii leżących w jednej
    płaszczyźnie, spotykających się w jednym punkcie i niebędących tą samą
    linią.
\item Gdy linie tworzące kąt są proste, taki kąt nazywa się prostoliniowym.
\item Kiedy dwie linie przecinają się w taki sposób, że przyległe kąty są
    sobie równe, to takie kąty są \bold{kątami prostymi}, a linie są wówczas
    do siebie \bold{prostopadłe}.
\item \bold{Kątem rozwartym} nazywa się kąt większy od kąta prostego.
\item Zaś \bold{kątem ostrym} nazywa się kąt mniejszy od kąta prostego.
\item Granicą jest to, co leży na skraju czegoś.
\item Granice tworzą \bold{figurę}.
\item \bold{Okrąg} jest to figura płaska, ograniczona jedną linią (zwaną
    obwodem) w taki sposób, że proste linie łączące jeden z punktów wewnątrz
    tej figury z granicą, są sobie równe.
\item A punkt ten jest nazywany \bold{środkiem} okręgu.
\item \bold{Średnica} okręgu jest to dowolna linia prosta łącząca dwa punkty
    na [granicy] okręgu i przechodząca przez jego środek. Każda średnica
    dzieli okrąg na dwie połowy.
\item \bold{Półokrąg} jest to połowa okręgu ograniczona średnicą wraz z tą
    średnicą. Środkiem półokręgu jest środek okręgu, z którego został
    wycięty.
\item \bold{Figury prostoliniowe} są to figury ograniczone wyłącznie
    prostymi liniami; \bold{trójkąty} są ograniczone trzema liniami,
    \bold{czworokąty}, czterem liniami, a \bold{wielokąty}, więcej niż
    czterema liniami.
\item \bold{Trójkąt równoboczny} to taki, którego wszystkie boki są równe;
    \bold{trójkąt równoramienny} to taki, którego dwa boki są równe,
    a \bold{różnoboczny} to taki, którego każdy bok jest innej długości.
\item \bold{Trójkąt prostokątny} to taki, który zawiera kąt prosty;
    \bold{trójkąt rozwartokątny} to taki, który zawiera kąt rozwarty,
    a \bold{ostrokątny} to taki, który ma wszystkie trzy kąty ostre.
\item \bold{Kwadrat} jest to czworokąt posiadające cztery kąty proste
    i wszystkie boki równej długości; \bold{prostokąt} posiada cztery
    kąty proste i boki parami równe; \bold{romb} ma wszystkie boki równej
    długości, a kąty różne od kątów prostych; \bold{równoległobok} zaś
    ma przeciwległe boki równej długości i kąty różne od kątów prostych.
    Pozostałe czworokąty nazywamy \bold{trapezami}.
\item \bold{Linie równoległe} to takie linie leżące w jednej płaszczyźnie,
    które przedłużane w nieskończoność w obu kierunkach, nie przecinają
    się w żadnym punkcie.
