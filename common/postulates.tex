\item Niech będzie uznane za fakt, że można narysować [dokładnie jedną]
    prostą linię z dowolnego punktu do dowolnego innego punktu\footnote{
        Słowa ,,dokładnie jedną'' nie znajdują się w tekście oryginalnym,
        jednak Euklides najwyraźniej zakłada, że może istnieć tylko jedna
        taka linia (patrz np. twierdzenie 1.4).
    }.
\item Oraz że tak narysowaną prostą linię można przedłużać w nieskończoność.
\item A także, że można narysować okrąg o dowolnym środku i promieniu,
\item Że wszystkie kąty proste są równe.
\item Oraz że jeżeli linia prosta przecinając dwie inne linie proste,
    tworzy po tej samej jej stronie wewnętrzne kąty, których suma jest
    mniejsza od dwóch kątów prostych, wówczas dwie pozostałe linie
    przedłużane w nieskończoność, przetną się po tej właśnie
    stronie\footnote{
        Postulat ten odróżnia geometrię płaską (euklidesową) od geometrii
        nieeukldesowych. O ile bowiem jest on spełniony na płaszczyźnie, o
        tyle na przykład na powierzchni kuli o tym, czy linie się przetną
        decyduje więcej czynników niż sama tylko suma kątów wewnętrznych.
    }.
