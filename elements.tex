\documentclass[12pt, a4paper]{scrartcl}
\usepackage[utf8]{inputenc}
\usepackage[T1]{fontenc}
\usepackage{amssymb}
\usepackage[polish]{babel}
\usepackage[margin=25mm]{geometry}
\usepackage{fancyhdr}
\usepackage{enumitem}
\usepackage{tikz}
\usepackage{hyperref}
\usepackage{style}

\title{Elementarz Starożytnych Greków}
\subtitle{Wybrane twierdzenia z ,,Elementów'' Euklidesa}
\author{Marcin Pastudzki}
\date{\today}

\makeatletter
\let\inserttitle\@title
\makeatother

\pagestyle{fancy}
\setlength{\headheight}{15pt}
\rhead{oprac. Marcin Pastudzki}
\lhead{\inserttitle}

\setlength{\parindent}{0pt}
\setlength{\parskip}{5mm}

\begin{document}
\maketitle

\part*{Rys historyczny}

Początki zajmowania się liczbami i geometrią w basenie Morza Śródziemnego wiąże
się przeważnie z Mezopotamią i Egiptem. Istotnie, tak Babilończycy, jak i
Egipcjanie potrafili obliczać różne rzeczy, szczególnie zaś pola i objętości
rozmaitych figur, a także prowadzili tablice astronomiczne. Nie zajmowali się
oni jednak matematyką dla samej matematyki --- przeciwnie, miała ona dla nich
charakter wysoce użytkowy.

Potrzebna była przede wszystkim inżynierom do wznoszenia budowli oraz urzędnikom
do inwentaryzacji magazynów, handlu czy rozdzielania pracy oraz dostępnych
zasobów. Z tego względu rachunki były tylko na tyle dokładne, aby budowle się
nie zawalały, opracowywano je na ogół metodą prób i błędów, a podawano w
formie przepisów, wyjaśniających krok po kroku, jak otrzymać żądany wynik.
Przyjmowano je tak długo, jak były wystarczająco skuteczne, bez żadnego
wyjaśnienia czy dowodu poprawności. Do tego stopnia było to dowolne i
nieformalne, że panuje opinia, wyrażona po raz pierwszy przez D. Knutha, że
bardziej niż z proto-matematyką mamy tu do czynienia z proto-informatyką, a
dokładniej z tworzeniem pierwszych algorytmów.

Za przykład niech posłuży okrąg. Patrząc nań, i rozumiejąc podobieństwo (które
to pojęcie wprowadzili jednak do geometrii dopiero Grecy), łatwo dojść do
przekonania, że długość okręgu powinna być proporcjonalna do jego średnicy
(i w konsekwencji -- promienia). Babilończycy również to rozumieli, a stosunek
długości okręgu do średnicy, który dziś oznaczamy grecką literą $\pi$,
oszacowali na liczbę $3$. Mówiąc dzisiejszym językiem, obliczali oni długość
okręgu ze wzoru: $l = 2 \cdot 3r$ (gdzie $r$ to promień okręgu). Spróbujmy
rozważyć konsekwencje tego faktu:

\begin{figure}[ht!]
    \begin{center}
      \begin{tikzpicture}
          \draw (0,0) circle (3);
          \draw (-3,0) -- (3, 0);
          \draw (-1.5,2.6) -- (1.5,-2.6);
          \draw (-1.5,-2.6) -- (1.5,2.6);
          \draw (-3,0) -- (-1.5,2.6) -- (1.5,2.6) -- (3,0) -- (1.5,-2.6) --
              (-1.5,-2.6) -- cycle;
          \draw (1.5,0) node [anchor=south] {$r$};
      \end{tikzpicture}
      \caption{Porównanie obwodu koła i sześciokąta foremnego.}
  \end{center}
\end{figure}

Ponieważ wielokąt foremny jest wpisany w okrąg, odcinki łączące środek
okręgu z wierzchołkami wielokąta tworzą równe kąty. Skoro tak, to każdy z nich
ma miarę: $\frac{360^\circ}{6} = 60^\circ$. Wynika z tego, że wszystkie
narysowane powyżej trójkąty są równoboczne, a zatem bok sześciokąta ma tę samą
długość co promień okręgu. W związku z tym, obwód sześciokąta ma długość $6r$,
tak samo jak okrąg, choć z rysunku gołym okiem widać, że obwód okręgu musi być
większy.

Z tabliczek babilońskich wynika, że ich autorzy byli świadomi tego faktu, jednak
w niczym nie przeszkadzało im to nadal posługiwać się przybliżeniem $\pi
\approx 3$, które to przybliżenie nawiasem mówiąc dla większości celów
związanych z budownictwem było wówczas zupełnie wystarczające.

Matematykę w takim sensie, w jakim dziś ją pojmujemy, wymyślili dopiero
starożytni Grecy, i to jak się wydaje, wymyślili ją jako jedyni w dziejach.
Wszystkie inne kultury na Ziemi, o ile zajmowały się w ogóle matematyką, albo
odziedziczyły ją po starożytnych Grekach, albo robiły to na sposób babiloński,
lub też, jak Hindusi i Arabowie, zajmowały się zagadkami logicznymi, jednak bez
żadnych prób traktowania ich w sposób naukowy lub wyciągania z nich wniosków na
temat otaczającego świata. Nieliczni tylko Arabowie, zresztą pod wpływem lektury
starożytnych Greków, zajmowali się matematyką w greckim sensie, wprowadzając przy
tym szereg użytecznych innowacji.

Podejście greckie było o tyle różne, że Grecy mieli ambicję osiągnąć wiedzę
pewną. Zamiast zatem obserwować świat i drogą indukcji budować zasady
nim rządzące (która to metoda w sposób oczywisty obarczona jest niepewnością
co do osiąganych rezultatów), przyjmowali arbitralne założenia i z nich drogą
dedukcji wyprowadzali prawa, które jako oparte na logicznym rozumowaniu, miały
charakter pewny, przynajmniej tak długo, jak spełnione były początkowe
założenia. Nie inaczej matematykę i w ogóle naukę (a przynajmniej niektóre jej
dyscypliny) uprawia się i dzisiaj.

Najdoskonalszy w starożytności wyraz ta dedukcyjna metoda dochodzenia do wiedzy
pewnej osiągnęła właśnie w ,,\emph{Elementach}'' Euklidesa. Aż do XIX w to
Euklides był dla matematyków wzorem tego, jak należy konstruować teorię
matematyczną, jak formułować aksjomaty, a następnie wyprowadzać z nich
twierdzenia. O samym autorze wiadomo niewiele. Żył Aleksandrii pod koniec IV w
p.n.e., w schyłkowym już okresie matematyki greckiej. Kierował tamtejszym
muzeum, którego częścią była słynna Biblioteka, jeden z Siedmiu Cudów Świata
Starożytnego. Większość dowiedzionych w ,,\emph{Elementach}'' twierdzeń była
znana Grekom już wcześniej, jednak zasługa Euklidesa polega na zebraniu ich
razem i oparciu na jednym, zwięzłym zestawie aksjomatów. W pewnym sensie jest to
apogeum greckiej myśli matematycznej, chociaż po Euklidesie tworzyli jeszcze
tacy znakomici matematycy jak Archimedes, którego prace nawiasem mówiąc mają z
reguły formę podobną do ,,\emph{Elementów}'', to znaczy zaczynają się od
definicji i postulatów, z których następnie wyprowadzane są kolejne twierdzenia.

Był to jednak już schyłek. Wraz z podbojem Grecji przez Rzymian rozwój
matematyki ustał praktycznie (pomijąc wkład nielicznych matematyków arabskich)
na okres ponad tysiąca lat i wznowiony został dopiero w okresie Odrodzenia.
Dobitnym, acz smutnym tego symbolem może być fakt, że oryginalny tekst
,,\emph{Elementów}'' (podobnie jak wielu innych dzieł starożytnych) nie dotrwał
do naszych czasów --- znamy to dzieło wyłącznie z przekładów na język arabski.

\pagebreak

\part{}
\subsection*{Definicje}

\begin{definitions}
    \item \bold{Punkt} jest tym, co niepodzielne.
    \item \bold{Linia} jest to długość bez szerokości.
    \item Linia jest ograniczona punktami.
    \item \bold{Prosta} jest to linia, położona równomiernie względem swoich
        punktów.
    \item \bold{Powierzchnia} posiada tylko długość i szerokość.
    \item Powierzchnia jest ograniczona liniami.
    \item \bold{Płaszczyzna} jest to powierzhnia położona równomiernie względem
        swoich punktów.
    \item \bold{Kąt płaski} jest to wzajemne nachylenie linii leżących w jednej
        płaszczyźnie, spotykających się w jednym punkcie i niebędących tą samą
        linią.
    \item Gdy linie tworzące kąt są proste, taki kąt nazywa się prostoliniowym.
    \item Kiedy dwie linie przecinają się w taki sposób, że przyległe kąty są
        sobie równe, to takie kąty są \bold{kątami prostymi}, a linie są wówczas
        do siebie \bold{prostopadłe}.
    \item \bold{Kątem rozwartym} nazywa się kąt większy od kąta prostego.
    \item Zaś \bold{kątem ostrym} nazywa się kąt mniejszy od kąta prostego.
    \item Granicą jest to, co leży na skraju czegoś.
    \item Granice tworzą \bold{figurę}.
    \item \bold{Okrąg} jest to figura płaska, ograniczona jedną linią (zwaną
        obwodem) w taki sposób, że proste linie łączące jeden z punktów wewnątrz
        tej figury z granicą, są sobie równe.
    \item A punkt ten jest nazywany \bold{środkiem} okręgu.
    \item \bold{Średnica} okręgu jest to dowolna linia prosta łącząca dwa punkty
        na [granicy] okręgu i przechodząca przez jego środek. Każda średnica
        dzieli okrąg na dwie połowy.
    \item \bold{Półokrąg} jest to połowa okręgu ograniczona średnicą wraz z tą
        średnicą. Środkiem półokręgu jest środek okręgu, z którego został
        wycięty.
    \item \bold{Figury prostoliniowe} są to figury ograniczone wyłącznie
        prostymi liniami; \bold{trójkąty} są ograniczone trzema liniami,
        \bold{czworokąty}, czterem liniami, a \bold{wielokąty}, więcej niż
        czterema liniami.
    \item \bold{Trójkąt równoboczny} to taki, którego wszystkie boki są równe;
        \bold{trójkąt równoramienny} to taki, którego dwa boki są równe,
        a \bold{różnoboczny} to taki, którego każdy bok jest innej długości.
    \item \bold{Trójkąt prostokątny} to taki, który zawiera kąt prosty;
        \bold{trójkąt rozwartokątny} to taki, który zawiera kąt rozwarty,
        a \bold{ostrokątny} to taki, który ma wszystkie trzy kąty ostre.
    \item \bold{Kwadrat} jest to czworokąt posiadające cztery kąty proste
        i wszystkie boki równej długości; \bold{prostokąt} posiada cztery
        kąty proste i boki parami równe; \bold{romb} ma wszystkie boki równej
        długości, a kąty różne od kątów prostych; \bold{równoległobok} zaś
        ma przeciwległe boki równej długości i kąty różne od kątów prostych.
        Pozostałe czworokąty nazywamy \bold{trapezami}.
    \item \bold{Linie równoległe} to takie linie leżące w jednej płaszczyźnie,
        które przedłużane w nieskończoność w obu kierunkach, nie przecinają
        się w żadnym punkcie.
\end{definitions}

\subsection*{Postulaty}

\begin{postulates}
    \item Niech będzie uznane za fakt, że można narysować [dokładnie jedną]
        prostą linię z dowolnego punktu do dowolnego innego punktu\footnote{
            Słowa ,,dokładnie jedną'' nie znajdują się w tekście oryginalnym,
            jednak Euklides najwyraźniej zakłada, że może istnieć tylko jedna
            taka linia (patrz np. twierdzenie 1.4).
        }.
    \item Oraz że tak narysowaną prostą linię można przedłużać w nieskończoność.
    \item A także, że można narysować okrąg o dowolnym środku i promieniu,
    \item Że wszystkie kąty proste są równe.
    \item Oraz że jeżeli linia prosta przecinając dwie inne linie proste,
        tworzy po tej samej jej stronie wewnętrzne kąty, których suma jest
        mniejsza od dwóch kątów prostych, wówczas dwie pozostałe linie
        przedłużane w nieskończoność, przetną się po tej właśnie
        stronie\footnote{
            Postulat ten odróżnia geometrię płaską (euklidesową) od geometrii
            nieeukldesowych. O ile bowiem jest on spełniony na płaszczyźnie, o
            tyle na przykład na powierzchni kuli o tym, czy linie się przetną
            decyduje więcej czynników niż sama tylko suma kątów wewnętrznych.
        }.
\end{postulates}

\begin{figure}[!h]
    \begin{center}
        \begin{tikzpicture}
            \draw (0,0) -- (10,2);
            \draw (0,4) -- (10,3);
            \draw (5,0) -- (5,4);
            \draw (5,1.5) arc (90:10:0.5);
            \draw (5,3) arc (-90:-5:0.5);
            \draw [dashed] (10,2) -- (15,3);
            \draw [dashed] (10,3) -- (15,2.5);
        \end{tikzpicture}
    \end{center}
\end{figure}

\subsection*{Ogólne założenia}

\begin{commons}
    \item Dwie rzeczy równe innej rzeczy, są także równe sobie nawzajem.
    \item Jeżeli do równych dodajemy równe, to otrzymane całości są równe.
    \item Jeżeli zaś od równych rzeczy odejmujemy równe, to pozostałości są
        nadal równe.
    \item Rzeczy pokrywające się wzajemnie są równe.
    \item Całość jest większa od części.
\end{commons}

\pagebreak

\section{}

Dany jest odcinek $AB$. Należy skonstruować \emph{trójkąt równoboczny} o
podstawie $AB$.

\begin{figure}[h!]
    \begin{center}
      \begin{tikzpicture}
          \draw (0,0) node [anchor=east] {A} -- (2,0) node [anchor=west] {B};
          \draw (2,0) circle (2);
          \draw (0,0) circle (2);
          \draw (-2,0) node [anchor=east] {D};
          \draw (4,0) node [anchor=east] {E};
          \draw (0,0) -- (1,1.73) node [anchor=south] {C} -- (2,0);
      \end{tikzpicture}
    \end{center}
\end{figure}

\begin{enumerate}
    \item Skonstruować okrąg o środku A i promieniu $AB$ \post{3}.
    \item Skonstruować okrąg o środku B i promieniu $AB$ \post{3}.
    \item Niech punkt przecięcia okręgów nazywa się $C$\footnote{
        Oczywiście są dwa takie punkty. Każdy z nich równie dobrze może posłużyć
        za punkt $C$.
    }.
    \item Odcinki $AB$, $AC$ i $BC$ tworzą trójkąt równoboczny \post{1}.
\end{enumerate}

\proof{
    Ponieważ punkty $A$ i $C$ leżą na tym samym okręgu o środku $B$, zatem $BA =
    BC$ \defin{1.15}.
    Podobnie punkty $B$ i $C$ leżą na tym samym okręgu o środku $A$, zatem $AB =
    AC$ \defin{1.15}.
    Zatem $AB = BC = AC$ \common{1}
}

\section{}

Możliwe jest skonstruowanie odcinka równego danemu odcinkowi w zadanym punkcie
(początkowym). Niech zadany punkt nazywa się $A$, zaś odcinek do powielenia
niech nazywa się $BC$.

\begin{enumerate}
    \item Skonstruować prostą linię $AB$ \post{1}.
    \item Skonstruować trójkąt równoboczny $ABD$ \prop{1.1}.
    \item Przedłużyć prostą linię $AD$ do punktu $E$ oraz $BD$ do punktu $F$
        \post{2}.
    \item Skonstruować okrąg o środku $B$ i promieniu $BC$ \post{3}.
    \item Niech punkt przecięcia tego okręgu z prostą $BF$ nazywa się $G$.
    \item Skostruować okrąg o środku $D$ i propmieniu $DG$ \post{3}.
    \item Punkt przecięcia tego okręgu z linią $DE$ niech nazywa się $H$.
\end{enumerate}

Linia $DH$ jest żądanym odcinkiem; jest on równy odcinkowi $BC$.

\pagebreak
\begin{figure}[h!]
    \begin{center}
        \begin{tikzpicture}
            \draw (1,1) node [anchor=west] {B} -- (1,3) node [anchor=south] {C};
            \draw (0,0) node [anchor=east] {A} -- (1,1) --
                (-0.36,1.36) node [anchor=south] {D} -- cycle;
            \draw (0,0) -- (0.77,-2.89) node [anchor=east] {E};
            \draw (1,1) -- (3.89,0.22) node [anchor=west] {F};
            \draw (1,1) circle (2);
            \draw (2.93,0.48) node [anchor=west] {G};
            \draw (-0.36,1.36) circle (3.41);
            \draw (0.51,-1.93) node [anchor=east] {H};
        \end{tikzpicture}
    \end{center}
\end{figure}

\proof{
    $BC$ i $BG$ są promieniami tego samego okręgu o środku $B$, zatem $BC = BG$
    \defin{1.15}. Z kolei $DG$ i $DH$ są promieniami tego samego okręgu o środku
    $D$, zatem $DG = DH$ \defin{1.15}. $DA = DB$, ponieważ są to boki trójkąta
    równobocznego \prop{1.1}. Skoro tak, to $AH = BG$ \common{3}. A zatem
    $AH = BG = BC$ \common{1}
}

\section{}

Gdy dane są dwa odcinki różnej długości, możeliwe jest odłożenie krótszego
odcinka na dłuższym. Niech dłuższy odcinek nazywa się $AB$, a krótszy -- $CD$.

\begin{figure}[h!]
    \begin{center}
        \begin{tikzpicture}
            \draw (5,3) node [anchor=south] {C} -- (8,3) node [anchor=south] {D};
            \draw (7,0) node [anchor=north] {B} -- (1,0) node [anchor=north] {A}
                -- (-1.12,2.12) node [anchor=east] {E};
            \draw (1,0) circle (3);
            \draw (4,0) node [anchor=north] {F};
        \end{tikzpicture}
    \end{center}
\end{figure}

\begin{enumerate}
    \item Skonstruować odcinek $AE$ równy odcinkowi $CD$ \prop{1.2}.
    \item Skonstruować okrąg o środku $A$ i promieniu $AE$ \post{3}.
    \item Niech punkt przecięcia tego okręgu z linią $AB$ nazywa się $F$.
\end{enumerate}

Otrzymany odcinek $AF$ jest równy odcinkowi $CD$.

\proof{
    Linie $AE$ i $AF$ są promieniami tego samego okręgu, zatem $AF = AE$. Linia
    $AE$ jest ponadto równa linii $CD$, co wynika z konstrukcji \prop{1.2}.
    Wobec tego $AF = AE = CD$ \common{3}
}

\section{}

Jeżeli trójkąt ma dwa boki równe odpowiednim bokom drugiego trójkąta, a kąty
zawarte pomiędzy tymi bokami są równe, to całe trójkąty są
\bold{przystające}\footnote{
    Euklides nie używa tego pojęcia.
}, to znaczy pozostałe boki oraz kąty pierwszego trójkąta muszą być równe
odpowiednim bokom i kątom drugiego trójkąta.

\begin{figure}[h!]
    \begin{center}
        \begin{tikzpicture}
            \draw (2.5,3) node [anchor=south] {A} -- (3,0) node [anchor=west] {C}
                -- (0,0) node [anchor=east] {B} -- cycle;
            \draw (7.5,3) node [anchor=south] {D} -- (8,0) node [anchor=west] {F}
                -- (5,0) node [anchor=east] {E} -- cycle;
        \end{tikzpicture}
    \end{center}
\end{figure}

Przyjmijmy zgodnie z tezą, że $AB = DE$, a $AC = DF$ oraz że
$\measuredangle BAC = \measuredangle EDF$.

\proof{
    Jeżeli przyłożymy $\triangle DEF$ do $\triangle ABC$\footnote{
        Euklides nigdzie nie wyjaśnia ani nie definiuje, co to znaczy przyłożyć
        figurę do figury. Jest to luka w rozumowaniu. W istocie należałoby
        możliwość przyłożenia figury do innej figury wraz z definicją dodać do
        postulatów.
    } tak, że punkt $D$ znajdzie się w miejscu punktu $A$, a bok $DE$ w miejscu
    boku $AB$, to punkt $E$ nałoży się na punkt $B$. A ponieważ $\measuredangle
    BAC = \measuredangle EDF$, to również bok $DF$ znajdzie się w miejscu boku
    $AC$, a punkt $F$ nałoży się na punkt $C$.

    W takim przypadku bok $EF$ musi nałożyć się na bok $BC$ \post{1}.
    Wobec tego boki te muszą być równe \common{4}. Również pozostałe kąty
    $\triangle DEF$ nałożą się na pozostałe kąty $\triangle ABC$, co oznacza,
    że te kąty także będą odpowiednio równe \common{4}
}

\end{document}
