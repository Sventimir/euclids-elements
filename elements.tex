\documentclass[12pt, a4paper]{article}
\usepackage[utf8]{inputenc}
\usepackage[T1]{fontenc}
\usepackage[polish]{babel}
\usepackage[margin=25mm]{geometry}
\usepackage{fancyhdr}
\usepackage{tikz}
\usepackage{style}

\title{Wybrane twierdzenia z ,,Elementów'' Euklidesa}
\author{Marcin Pastudzki}
\date{\today}

\makeatletter
\let\inserttitle\@title
\makeatother

\pagestyle{fancy}
\setlength{\headheight}{15pt}
\rhead{oprac. Marcin Pastudzki}
\lhead{\inserttitle}

\setlength{\parindent}{0pt}
\setlength{\parskip}{5mm}

\begin{document}

\section*{Rys historyczny}

Początki zajmowania się liczbami i geometrią w basenie Morza Śródziemnego wiąże
się przeważnie z Mezopotamią i Egiptem. Istotnie, tak Babilończycy, jak i
Egipcjanie potrafili obliczać różne rzeczy, szczególnie zaś pola i objętości
rozmaitych figur, a także prowadzili tablice astronomiczne. Nie zajmowali się
oni jednak matematyką dla samej matematyki --- przeciwnie, miała ona dla nich
charakter wysoce użytkowy.

Potrzebna była przede wszystkim inżynierom do wznoszenia budowli oraz urzędnikom
do inwentaryzacji magazynów, handlu czy rozdzielania pracy oraz dostępnych
zasobów. Z tego względu rachunki były tylko na tyle dokładne, aby budowle się
nie zawalały, opracowywano je na ogół metodą prób i błędów, a podawano w
formie przepisów, wyjaśniających krok po kroku, jak otrzymać żądany wynik.
Przyjmowano je tak długo, jak były wystarczająco skuteczne, bez żadnego
wyjaśnienia czy dowodu poprawności. Do tego stopnia było to dowolne i
nieformalne, że panuje opinia, wyrażona po raz pierwszy przez D. Knutha, że
bardziej niż z proto-matematyką mamy tu do czynienia z proto-informatyką, a
dokładniej z tworzeniem pierwszych algorytmów.

Za przykład niech posłuży okrąg. Patrząc nań, i rozumiejąc podobieństwo (które
to pojęcie wprowadzili jednak do geometrii dopiero Grecy), łatwo dojść do
przekonania, że długość okręgu powinna być proporcjonalna do jego średnicy
(i w konsekwencji -- promienia). Babilończycy również to rozumieli, a stosunek
długości okręgu do średnicy, który dziś oznaczamy grecką literą $\pi$,
oszacowali na liczbę $3$. Mówiąc dzisiejszym językiem, obliczali oni długość
okręgu ze wzoru: $l = 2 \cdot 3r$ (gdzie $r$ to promień okręgu). Spróbujmy
rozważyć konsekwencje tego faktu:

\begin{figure}[h!]
    \begin{center}
      \begin{tikzpicture}
          \draw (0,0) circle (3);
          \draw (-3,0) -- (3, 0);
          \draw (-1.5,2.6) -- (1.5,-2.6);
          \draw (-1.5,-2.6) -- (1.5,2.6);
          \draw (-3,0) -- (-1.5,2.6) -- (1.5,2.6) -- (3,0) -- (1.5,-2.6) --
              (-1.5,-2.6) -- cycle;
          \draw (1.5,0) node [anchor=south] {$r$};
      \end{tikzpicture}
      \caption{Porównanie obwodu koła i sześciokąta foremnego.}
  \end{center}
\end{figure}

Ponieważ wielokąt foremny jest wpisany w okrąg, przeto odcinki łączące środek
okręgu z wierzchołkami wielokąta tworzą równe kąty. Skoro tak, to każdy z nich
ma miarę: $\frac{360^\circ}{6} = 60^\circ$. Wynika z tego, że wszystkie
narysowane powyżej trójkąty są równoboczne, a zatem bok sześciokąta ma tę samą
długość co promień okręgu. W związku z tym, obwód sześciokąta ma długość $6r$,
tak samo jak okrąg, choć z rysunku gołym okiem widać, że obwód okręgu musi być
większy.

Z tabliczek babilońskich wynika, że ich autorzy byli świadomi tego faktu, jednak
w niczym nie przeszkadzało im to nadal posługiwać się przybliżeniem $\pi
\approx 3$, które to przybliżenie nawiasem mówiąc dla większości celów
związanych z budownictwem jest zupełnie wystarczające.

Matematykę w takim sensie, w jakim dziś ją pojmujemy, wymyślili dopiero
starożytni Grecy, i to jak się wydaje, wymyślili ją jako jedyni w dziejach.
Wszystkie inne kultury na Ziemi, o ile zajmowały się w ogóle matematyką, albo
odziedziczyły ją po starożytnych Grekach, albo robiły to na sposób babiloński,
lub też, jak Hindusi i Arabowie, zajmowały się zagadkami logicznymi, jednak bez
żadnych prób traktowania ich w sposób naukowy lub wyciągania z nich wniosków na
temat otaczającego świata. Nieliczni tylko Arabowie, zresztą pod wpływem lektury
starożytnych Greków, zajmowali się matematyką w greckim sensie, choć wprowadzili
oni szereg użytecznych innowacji.

Podejście greckie było o tyle różne, że Grecy mieli ambicję osiągnąć wiedzę
pewną. Zamiast zatem obserwować świat i drogą indukcji budować zasady
nim rządzące (która to metoda w sposób oczywisty obarczona jest niepewnością
co do osiąganych rezultatów), przyjmowali arbitralne założenia i z nich drogą
dedukcji wyprowadzali prawa, które jako oparte na logicznym rozumowaniu, miały
charakter pewny, przynajmniej tak długo, jak spełnione były początkowe
założenia. Nie inaczej matematykę i w ogóle naukę (a przynajmniej niektóre jej
dyscypliny) uprawia się i dzisiaj.

Najdoskonalszy w starożytności wyraz ta dedukcyjna metoda dochodzenia do wiedzy
pewnej osiągnęła właśnie w ,,\emph{Elementach}'' Euklidesa. Aż do XIX w to
Euklides był dla matematyków wzorem tego, jak należy konstruować teorię
matematyczną, jak formułować aksjomaty, a następnie wyprowadzać z nich
twierdzenia. O samym autorze wiadomo niewiele. Żył pod koniec IV w p.n.e.,
w schyłkowym już okresie matematyki greckiej. Większość dowiedzionych w
,,\emph{Elementach}'' twierdzeń była znana Grekom już wcześniej, jednak zasługa
Euklidesa polega na zebraniu ich razem i oparciu na jednym, zwięzłym zestawie
aksjomatów. W pewnym sensie jest to apogeum greckiej myśli matematycznej,
chociaż po Euklidesie tworzyli jeszcze tacy znakomici matematycy jak Archimedes,
którego prace nawiasem mówiąc mają z reguły formę podobną do ,,\emph{Elementów}''.

Był to jednak już schyłek. Wraz z podbojem Grecji przez Rzymian rozwój
matematyki ustał praktycznie (pomijąc wkład nielicznych matematyków arabskich)
na okres ponad tysiąca lat i wznowiony został dopiero u schyłku Średniowiecza.
Dobitnym, acz smutnym tego symbolem może być fakt, że oryginalny tekst
,,\emph{Elementów}'' (podobnie jak wielu innych dzieł starożytnych) nie dotrwał
do naszych czasów --- znamy to dzieło wyłącznie z przekładów na język arabski.

\pagebreak

\section*{Księga I}
\subsection*{Twierdzenie I}

Dany jest odcinek $AB$. Należy skonstruować \emph{trójkąt równoboczny} o
podstawie $AB$.

\begin{figure}[h!]
    \begin{center}
      \begin{tikzpicture}
          \draw (0,0) node [anchor=east] {A} -- (2,0) node [anchor=west] {B};
          \draw (2,0) circle (2);
          \draw (0,0) circle (2);
          \draw (-2,0) node [anchor=east] {D};
          \draw (4,0) node [anchor=east] {E};
          \draw (0,0) -- (1,1.73) node [anchor=south] {C} -- (2,0);
      \end{tikzpicture}
    \end{center}
\end{figure}

\begin{enumerate}
    \item Narysować okrąg o środku A i promieniu $|AB|$ \post{3}.
    \item Narysować okrąg o środku B i promieniu $|AB|$ \post{3}.
    \item Niech punkt przecięcia okręgów nazywa się $C$\footnote{
        Oczywiście są dwa takie punkty. Każdy z nich równie dobrze może posłużyć
        za punkt $C$.
    }.
    \item Odcinki $AB$, $AC$ i $BC$ tworzą trójkąt równoboczny \post{1}.
\end{enumerate}

\proof {
    Ponieważ punkty $A$ i $C$ leżą na tym samym okręgu o środku $B$, zatem $|BA| =
    |BC|$ \defin{1.15}.
    Podobnie punkty $B$ i $C$ leżą na tym samym okręgu o środku $A$, zatem $|AB| =
    |AC|$ \defin{1.15}.
    Zatem $|AB| = |BC| = |AC|$ \axiom{1}.
}

\end{document}
