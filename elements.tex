\documentclass[12pt, a4paper]{article}
\usepackage[utf8]{inputenc}
\usepackage[T1]{fontenc}
\usepackage[polish]{babel}
\usepackage[margin=25mm]{geometry}
\usepackage{fancyhdr}
\usepackage{tikz}
\usepackage{style}

\title{Wybrane twierdzenia z ,,Elementów'' Euklidesa}
\author{Marcin Pastudzki}
\date{\today}

\makeatletter
\let\inserttitle\@title
\makeatother

\pagestyle{fancy}
\setlength{\headheight}{15pt}
\rhead{oprac. Marcin Pastudzki}
\lhead{\inserttitle}

\setlength{\parindent}{0pt}
\setlength{\parskip}{5mm}

\begin{document}

\section*{Księga I}
\subsection*{Twierdzenie I}

Dany jest odcinek $AB$. Należy skonstruować \emph{trójkąt równoboczny} o
podstawie $AB$.

\begin{figure}[h!]
    \begin{center}
      \begin{tikzpicture}
          \draw (0,0) node [anchor=east] {A} -- (2,0) node [anchor=west] {B};
          \draw (2,0) circle (2);
          \draw (0,0) circle (2);
          \draw (-2,0) node [anchor=east] {D};
          \draw (4,0) node [anchor=east] {E};
          \draw (0,0) -- (1,1.73) node [anchor=south] {C} -- (2,0);
      \end{tikzpicture}
    \end{center}
\end{figure}

\begin{enumerate}
    \item Narysować okrąg o środku A i promieniu $|AB|$ (post. 3.).
    \item Narysować okrąg o środku B i promieniu $|AB|$ (post. 3.).
    \item Niech punkt przecięcia okręgów nazywa się $C$.
    \item Odcinki $AB$, $AC$ i $BC$ tworzą trójkąt równoboczny (post. 1.).
\end{enumerate}

\textbf{Dowód}:

Ponieważ punkty $A$ i $C$ leżą na tym samym okręgu o środku $B$, zatem $|BA| = |BC|$.
Podobnie punkty $B$ i $C$ leżą na tym samym okręgu o środku $A$, zatem $|AB| = |AC|$.
Zatem $|AB| = |BC| = |AC|$, \emph{q.e.d.}
\end{document}
